\chapter{Conclusiones y Discusiones}

A lo largo del presente trabajo se mostraron una serie de variantes a los métodos para analizar señales fotofísicas provenientes de experimentos basados en dinámica de sensores de homoFRET. Entre estos métodos, se mostró la mejor forma de transformar los observables experimentales (intensidades cruzadas) para recuperar el estado del ensamble del fluoróforo instante a instante. A continuación, se presentó un modelo matemático, con su correspondiente adaptación, que se utilizó para estudiar el comportamiento de los elementos del sistema biológico. A partir de este, fue posible diseñar un análisis que permite obtener el estado de la variable biológica de interés haciendo uso únicamente de la información provista por el sensor.

Luego de unir el desarrollo realizado en métodos de análisis de datos experimentales con el correspondiente al modelado del sistema biológico se procedió a analizar datos provenientes de un experimento. La aplicación de estos métodos a un experimento real sirvió como prueba de concepto de los métodos presentados.

%\section{Conclusión}

En primer lugar, se detallaron las razones por las que un análisis robusto de la señal de anisotropía debe comprender una transformación de los datos experimentales en información sobre el ensamble de fluoróforos. Debido a que el brillo de un monómero en estado asociado (dímero) o disociado (monómero) es distinto, es necesario utilizar ambos observables experimentales para estudiar el estado del ensamble de fluoróforos. Esta diferencia en brillos no se debe únicamente a diferencias en la detección de los distintos fluoróforos, ya sea por diferencias en emisión, rendimiento cuántico o distinta respuesta del canal de detección para longitudes de onda distintas, sino que también pueden darse diferencias en la maduración de los fluoróforos apareados. Al ocurrir transferencia de energía entre fluoróforos, la intensidad de fluorescencia detectada también variará, incluso cuando la cantidad de fluoróforoso de cada especie sea la misma.

%cuando estos se hallan asociados y hay transferencia de energía entre ellos, el canal de detección detecta a ambos con el mismo brillo ya que sus diferencias en espectro, combinados con los filtros del sistema óptico, generan monómeros con brillos distintos.

%Es usual hallar en la bibliografía que se ajustan las curvas de anisotropía utilizando funciones sigmoideas y se toman como parámetros relevantes la velocidad de transición, así como el tiempo en que se alcanza la mitad del rango de anisotropía. Como se mostró en el capítulo \ref{cap:microscopia}, este método no debe utilizarse cuando el fluoróforo presenta brillos distintos según este asociado o disociado.

%siempre resulta el más conveniente debido a que se trunca el análisis a nivel de la señal fotofísica, sin indagar en el estado del ensamble de fluoróforos, ni en el estado del sistema biológico en estudio.



%Una vez desarrollado el método para transformar las intensidades cruzadas en información del estado del ensamble de fluoróforos y el 

En el capítulo \ref{cap:microscopia}, se mostraron métodos que ajustaban tanto los observables directos, intensidad paralela y perpendicular, como un método que utilizaba una combinación de estos observables que eran la anisotropía y la fluorescencia total. En ambos métodos, era necesario ajustar un parámetro de escala para la intensidad de fluorescencia. Posteriormente, se mostró que ajustarlo trae aparejado el inconveniente que diferencias en la segmentación de las imágenes causan aumentos repentinos de las intensidades que complican el ajuste de los datos experimentales. Con el objetivo de corregir este problema, se planteó como solución ajustar las intensidades cruzadas normalizadas por la intensidad total del sistema.

Ajustar las curvas de intensidades cruzadas normalizadas nos provee ventajas desde varias perspectivas distintas. Teóricamente, elimina el parámetro de escala de todos los ajustes. Desde el punto de vista experimental, este método elimina efectos generados por problemas en la segmentación de imágenes para analizar. Por último, provee la transformación más fiel de los datos experimentales a la curva que determina el estado del ensamble de fluoróforos.

%1. Teoricamente nos sacamos de encima parametro de escala
%2. EN simulaciones es el mas fiel para ajustar la anisotropía, pero traduce todo el ruido en m
%3. En datos reales lo mostramos con el experimento 3sensores:1caspasa, sale de la regresion lineal, ademas de la explicacion de como nos desentendemos de los problemas  de segmentacion de imagenes

Entre los problemas que presentó el ajuste de las intensidades cruzadas observadas se encuentra el desconocimiento de muchos de los parámetros involucrados. Por un lado, el parámetro $b$ que relaciona el brillo de un monómero en estado asociado con el brillo del monómero disociado puede calcularse experimentalmente mediante experimentos de \ening{N\&B}. En segundo lugar, se recomienda generar células que expresen biosensores no clivables, así como otro grupo de células que expresen sensores clivados en su totalidad para obtener mediciones de las anisotropías máxima y mínima en que puede variar la anisotropía de un par de fluoróforos. Como se explicó previamente, no alcanza con determinar estos valores teóricamente ya que se observaron errores sistemáticos distintos en las anisotropías máximas y mínimas observadas experimentalmente.

En el capítulo \ref{cap:modelo}, se adaptó el modelo matemático desarrollado por \textit{Sorger et al.}\cite{Sorger2008} para que describa adecuadamente el comportamiento de los sensores transfectados a la célula. Se concluyó el capítulo mostrando diversas técnicas numéricas que permitían identificar el estado de variables de interés biológico. Estas variables son la cantidad de complejo caspasa:sensor que esta íntimamente relacionada con la actividad de la caspasa en cada instante, y la proporción de caspasa activa. Por otro lado, debemos recalcar que estos métodos numéricos no pudieron ser aplicados a todas las curvas. Para contrarrestar estos inconvenientes se recomienda mejorar la resolución temporal, con el objetivo de facilitar el cálculo de la derivada temporal de la curva de proporción de sensor en estado monomérico; y por otro lado, utilizar los experimentos control mencionados previamente para confeccionar un mejor filtrado de los puntos en que la curva de proporción de sensor en estado dimérico es muy baja.

Con el objetivo de obtener información sobre el estado de la caspasa a partir de la curva de proporción de monómero, se plantearon ajustes de esta curva a partir de simulaciones de cinética enzimática. A continuación, se validó dicho método mediante su aplicación a un experimento control. Este método fue útil para obtener las curvas de proporción de caspasa activa de las curvas experimentales en estudio para luego estimar los tiempos de activación de cada caspasa. Estos resultados fueron contrastados posteriormente con el orden de activación predicho por el modelo teórico donde se apreció una elevada concordancia para el tiempo de activación de caspasa al 10$\%$. Las diferencias en separación temporal o en orden de activación al 50$\%$ pueden corregirse si se varían las condiciones iniciales del sistema.

En conclusión, se presentó en este trabajo un análisis exhaustivo de los datos experimentales que permite transformar el observable fotofísico, que son las intensidades cruzadas, en la proporción de caspasa activa, que representa la variable de interés biológico. Por otro lado, el modelo matemático adaptado no solo sirve para contrastar los datos experimentales observados, sino que predice cambios en el orden de activación de las caspasas ante distintos estímulos. Esto último, es el puntapié inicial para evaluar la respuesta del mismo sistema celular ante distintos estímulos y estudiar las diferencias en el orden de activación de las caspasas.