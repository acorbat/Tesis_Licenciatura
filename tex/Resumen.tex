\begin{abstract}
    Comprender la cadena de eventos que induce a las células a iniciar un programa de muerte celular es esencial para dilucidar el origen de la señal que lo estimula. Para lograrlo, es necesario poder estudiar simultáneamente la dinámica de varias especies involucradas en el proceso. Esto último es posible mediante la utilización de biosensores de las vías iniciadoras y efectoras de la apoptosis basados en homoFRET. El objetivo del trabajo fue desarrollar métodos analíticos exhaustivos en la obtención de información a partir de experimentos en que se induce apoptosis en células.
    
    %Para lograrlo, se adaptó un modelo matemático basado en ley de acción de masas y qque describe la dinámica del sistema
    
    %Las caspasas pertenecen a una familia de proteínas denominado cisteín-proteasas y se encargan de seccionar otras proteínas reconociendo una secuencia específica de aminoácidos. Dichas proteasas conforman los engranajes principales de la maquinaria apoptótica, una forma programada de muerte celular. El objetivo del presente trabajo fue el de adaptar un modelo matemático basado en ley de acción de masas y ecuaciones diferenciales acopladas que permita comprender y predecir el comportamiento de la cascada de señalización de las caspasas. 
    
    Se generó una batería de métodos que permiten obtener información del ensamble de fluoróforos a partir de los observables fotofísicos (intensidades cruzadas). El mejor candidato para realizar esta transformación es un método de ajuste sobre las curvas de intensidades cruzadas normalizadas por la intensidad total. Cabe destacar que esta transformación permite hallar el estado del ensamble de fluoróforos instante a instante con una mínima pérdida de información.
    
    %Con el objetivo de contrastar experimentalmente el modelo matemático adaptado, se analizaron datos experimentales provenientes de estudios de células durante apoptosis basados en homoFRET. Para ello fue necesario generar una batería de métodos que permitiesen obtener información de un ensamble de fluoróforos a partir de los observables fotofísicos (intensidades cruzadas). El mejor candidato para realizar esta transformación es un método de ajuste sobre las curvas de intensidades cruzadas normalizadas por la intensidad total.
    
    A continuación, se muestra que si se conoce la proporción de sensor en estado monomérico, su derivada temporal  ($\Delta m$) es útil para despejar la cantidad de caspasa en complejo con el sensor, es decir, en actividad. Simultáneamente, se aprecia que si a $\Delta m$ lo dividimos por la cantidad de sensor sin clivar ($\Delta m/d$) obtenemos directamente la proporción de caspasa activa, es decir, nuestra variable de interés. Dado que estos métodos presentan dificultades numéricas, se plantea y valida como alternativa un ajuste basado en la simulación de la cinética enzimática de caspasa.
    
    En conclusión, se presentó en este trabajo un análisis exhaustivo de los datos experimentales que permite transformar el observable fotofísico, que son las intensidades cruzadas, en la proporción de caspasa activa, que representa la variable de interés biológico. Por otro lado, el modelo matemático adaptado no solo sirve para contrastar los datos experimentales observados, sino que predice cambios en el orden de activación de las caspasas ante distintos estímulos. En conjunto, estos métodos proveen una herramienta para estudiar la propagación de la señal en la red apoptótica y así determinar su origen.
\end{abstract}